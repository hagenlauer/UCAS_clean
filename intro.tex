\section{Introduction}
Cloud Computing has become ubiquitous in a plethora of applications ranging from education, finance, and smart home to healthcare, government, and military applications~\cite{cloudview,edge,fog,virtiot}. However, the cloud-centered paradigm faces a major shift: the success of the Internet-of-Things causes the generation of the majority of data at the outer edges of a network~\cite{edge,edgepromise,edge2}. \emph{Edge computing} refers to the set of technologies allowing computations to be performed along the edges of the network, on downstream data on behalf of cloud services and upstream data on behalf of IoT services~\cite{edge,edge2}. \emph{Fog computing} is closely related to the general concept of \emph{edge computing}\cite{fog, fog1, fog2} with a strong focus on performing task in nearby, decentralized systems. For some tasks, this yields considerable feats such as lower latency and improved user-experience as well as resilience through redundancy for services. Virtualization, essential for concepts ranging from \emph{Infrastructure-as-a-Service} to \emph{Functions-as-a-Service} implemented by vendors like Microsoft, Google, and Amazon has fundamentally changed the way software and data is being handled and is the backbone of modern computing infrastructure, especially with an increase in service decentralization and dedicated, collaborating nodes\cite{virtiot,edge,fog}.\\
The importance of trust and trust establishment strategies becomes apparent in decentralized systems with no immediately recognizable authorities. The rather ambiguous issue of trust and collaboration can be demonstrated using a very small, discrete example:\\
Suppose a mathematician who is interested in number theory uses a computer with a program for factorizing numbers. The output that will be produced by that program is either the factorization of a given number or a statement that the given number is a prime. Now suppose that the same mathematician wishes to inspect a large number, too large to verify without the aid of the computer. The mathematician can have two possible expectations at this point: the given number is a prime number or not. Assuming there are strong reasons to believe that the number is a prime, the result of the program can either confirm this by telling that the number is a prime or give the factorization as evidence that intuition has in fact fooled the mathematician. The situation changes, however, if the mathematician has strong reasons to believe that the given number is not prime. Again, the computer can produce two possible outputs: the number is a prime or a factorization. If the output is a factorization, the mathematician can confirm the belief by recalculating the given number. However, if the computer comes back with the result that the number is a prime, contrary to strong reasons leading one to believe otherwise, why should the mathematician trust this result?~\cite{Dijkstra1979} This example illustrates that even in completely discrete problems, the computation may not be worthwhile if it lacks convincing power w.r.t. the \emph{quality} of the result. As possibly dated and oversimplified as it may seem, the issue raised here, instead of being remedied, is being amplified by modern efforts using more flexible, decentralized computing systems. As a practical set of standards-based technologies, Trusted Computing~\cite{ISOTPM} can serve to supply evidence about a computing platform. The process of collecting, supplying, and appraising evidence, and ultimately a result, is referred to as Remote Attestation~\cite{RA,DAA,lauer2016}. 

\subsection{Contribution}
This paper introduces current technological and standardization efforts towards trustworthy cloud computations. Implementing trustworthy virtualized systems currently requires the adoption of at least two standards for hardware and application level trust. We outline challenges and potential conflicts related to translating \emph{trust} across such standards. We then review and evaluate current trust establishment methods and put them into perspective of decentralized systems. Finally, we propose user-centered attestation as a candidate for layered, decentralized systems along with a methodology and strategy for specifying and synthesizing such an attestation system.
